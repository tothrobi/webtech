\chapter*{Forráskód}
\addcontentsline{toc}{chapter}{Forráskód} 
\section*{Nézetek} 
\addcontentsline{toc}{section}{Nézetek}
\subsection*{WelcomeView nézet}
\addcontentsline{toc}{subsection}{WelcomeView nézet}
\begin{spacing}{2}
\end{spacing}

\begin{minipage}{\textwidth}
    \linespread{0.8}\selectfont
    \begin{lstlisting}[language=swift]
        struct WelcomeView: View {
            @State private var isShowingLoginView = false
            @State private var isShowingRegisterView = false
            @EnvironmentObject var viewRouter: ViewRouter
            
            var body: some View {
                    ZStack{
                        SetBackgroundImage(image: "Background1")
                        VStack{
                            HStack {
                                Text("Fitness is the")
                                    .font(.custom(.light, size: 20))
                                    .foregroundColor(.white)
                                Text("KEY")
                                    .font(.custom(.bold, size: 25))
                                    .foregroundColor(Color.theme.lighBlue)
                            }
                            .padding(.top,70)
                      
                            Image("Logo")
                           
                            VStack{
                                Button(action: {
                                    withAnimation {
                                        viewRouter.currentPage = .logInPage
                                    }
                                }) {
                                    PrimaryButton(title: "Login", fontColor: Color.theme.white, backgroundColor: Color.theme.lighBlue)
                                }
                                Button(action: {
                                    withAnimation {
                                        viewRouter.currentPage = .registerPage
                                    }
                                }) {
                                    PrimaryButton(title: "Registration", fontColor: Color.theme.lighBlue, backgroundColor: Color.theme.darkGray)
                                        .overlay(RoundedRectangle(cornerRadius: 5).strokeBorder(Color.theme.lighBlue, lineWidth: 1))
                                }
                            }
                            .frame(width: UIScreen.main.bounds.width - 32)
                            .padding(.bottom,70)
                        }
                }
            }
        }
        struct WelcomeView_Previews: PreviewProvider {
            static var previews: some View {
                NavigationView {
                    WelcomeView()
                        .navigationBarHidden(true)
                }
            }
        }
    \end{lstlisting}   
\end{minipage}


\subsection*{LoginView nézet}
\addcontentsline{toc}{subsection}{Login nézet}

\begin{spacing}{2}
\end{spacing}
\begin{minipage}{\textwidth}
    \linespread{0.8}\selectfont
    \begin{lstlisting}[language=swift]
        struct LoginView: View {
            @ObservedObject var viewModel = ViewModel()
            @EnvironmentObject var viewRouter: ViewRouter
            
            @State var loginProcessing = false
            @State private var showingAlert = false
            
            @State var loginErrorMessage = ""
            
            @State private var email = ""
            @State private var password = ""
            
            var dataIsValid: Bool {
                   return !loginProcessing && !email.isEmpty && !password.isEmpty
               }
            var buttonColor: Color {
                return dataIsValid ? Color.theme.lighBlue : .gray
            }
            
            
                NavigationView {
                    ZStack{
                        SetBackgroundImage(image: "Background2")
                        
                        VStack(alignment: .leading){
                            HeaderView(title: "Login", navigationPage: .welcomePage, arrowVisibility: true)
                           
                            VStack(alignment: .leading, spacing: 12){
                                Text("Hi User!")
                                    .font(.custom(.regular, size: 20))
                                    .foregroundColor(Color.theme.lighBlue)
        
                                Text("Log in to continue.")
                                    .font(.custom(.light, size: 16))
                                    .foregroundColor(Color.theme.white)
                            }
                            .padding(.top, 50)
                            
                            VStack{
                                TextFieldView(text: $email,title: "Email")
                                
                                SecureFieldView(text: $password, title: "Password")
                            }
                            Spacer()
                            VStack(spacing: 15){
                                Button(action: {
                                    showingAlert = true
                                    loginUser(email: email, password: password)
                                }) {
                                    PrimaryButton(title: "Login", fontColor: Color.theme.white, backgroundColor: buttonColor)
                                        .padding(.top)
                                }
                                .disabled(!dataIsValid)
                                if loginProcessing {
                                    ProgressView()
                                }
                                if !loginErrorMessage.isEmpty {
                                    Text("Failed login: \(loginErrorMessage)")
                                        .foregroundColor(.red)
                                }
                            }
                            Text("or sign in using socialmedia")
                        .font(.custom(.light, size: 16))
                        .foregroundColor(Color.theme.white)
                        .tracking(0.7)
                        .frame(maxWidth: .infinity,alignment: .center)
                        .padding(.vertical, 10)
                    
                    VStack(spacing: 15){
                        SocialLoginButton(image: "AppleLogo", text: "Sing in with AppleID")
                            
                        SocialLoginButton(image: "GoogleLogo", text: "Sign in with Google")
                    }
                
                }
                
    \end{lstlisting}   
\end{minipage}

\hspace{-10mm}
\begin{minipage}{\textwidth}
    \linespread{0.8}\selectfont
    \begin{lstlisting}[language=swift]
        struct LoginView: View {
            @ObservedObject var viewModel = ViewModel()
            @EnvironmentObject var viewRouter: ViewRouter
            
            @State var loginProcessing = false
            @State private var showingAlert = false
            
            @State var loginErrorMessage = ""
            
            @State private var email = ""
            @State private var password = ""
            
            var dataIsValid: Bool {
                return !loginProcessing && !email.isEmpty && !password.isEmpty
            }
            var buttonColor: Color {
                return dataIsValid ? Color.theme.lighBlue : .gray
            }
    
            var body: some View {
                NavigationView {
                    ZStack{
                        SetBackgroundImage(image: "Background2")
                        
                        VStack(alignment: .leading){
                            HeaderView(title: "Login", navigationPage: .welcomePage, arrowVisibility: true)
                        
                            VStack(alignment: .leading, spacing: 12){
                                Text("Hi User!")
                                    .font(.custom(.regular, size: 20))
                                    .foregroundColor(Color.theme.lighBlue)

                                Text("Log in to continue.")
                                    .font(.custom(.light, size: 16))
                                    .foregroundColor(Color.theme.white)
                            }
                            .padding(.top, 50)
                            
                            VStack{
                                TextFieldView(text: $email,title: "Email")
                                
                                SecureFieldView(text: $password, title: "Password")
                            }
                            Spacer()
                            VStack(spacing: 15){
                                Button(action: {
                                    showingAlert = true
                                    loginUser(email: email, password: password)
                                }) {
                                    PrimaryButton(title: "Login", fontColor: Color.theme.white, backgroundColor: buttonColor)
                                        .padding(.top)
                                }
                                .disabled(!dataIsValid)
                                if loginProcessing {
                                    ProgressView()
                                }
                                if !loginErrorMessage.isEmpty {
                                    Text("Failed login: \(loginErrorMessage)")
                                        .foregroundColor(.red)
                                }
                            }
                            
                            Text("or sign in using socialmedia")
                                .font(.custom(.light, size: 16))
                                .foregroundColor(Color.theme.white)
                                .tracking(0.7)
                                .frame(maxWidth: .infinity,alignment: .center)
                                .padding(.vertical, 10)
                            
                            VStack(spacing: 15){
                                SocialLoginButton(image: "AppleLogo", text: "Sing in with AppleID")
                                    
                                SocialLoginButton(image: "GoogleLogo", text: "Sign in with Google")
                            }
                    }
                }
                .navigationBarHidden(true)      
            }
        }
                
    \end{lstlisting}   
\end{minipage}

\hspace{-10mm}
\begin{minipage}{\textwidth}
    \linespread{0.8}\selectfont
    \begin{lstlisting}[language=swift]
        HStack{
                                Image("Logo2")
                                    .padding(.vertical, 40)
                                    .frame(maxWidth: .infinity,alignment: .center)
                                    .shadow(color: Color.theme.darkGray.opacity(0.25), radius: 5, x: 0, y: 4)
                            }
                        }
                        .frame(width: UIScreen.main.bounds.width - 52)
        func loginUser(email: String, password: String) {
                        
                        loginProcessing = true
                        Auth.auth().signIn(withEmail: email, password: password) { authResult, error in
                            
                            guard error == nil else {
                                loginProcessing = false
                                loginErrorMessage = error!.localizedDescription
                                return
                            }
                            switch authResult {
                            case .none:
                                print("Could not sign in user.")
                                loginProcessing = false
                            case .some(_):
                                print("User signed in")
                                loginProcessing = false
                                withAnimation {
                                    viewRouter.currentPage = .homePage
                                }
                            }
                            
                        }
                    }
                    
        }
        struct LoginView_Previews: PreviewProvider {
            static var previews: some View {
                LoginView()
            }
        }
    \end{lstlisting}   
\end{minipage}


\subsection*{RegisterView nézet}
\addcontentsline{toc}{subsection}{RegisterView nézet}

\begin{spacing}{2}
\end{spacing}
\begin{minipage}{\textwidth}
    \linespread{0.8}\selectfont
    \begin{lstlisting}[language=swift]
        struct RegisterView: View {
            @ObservedObject var viewModel = ViewModel()
            @EnvironmentObject var viewRouter: ViewRouter
            
            @State var registerProcessing = false
            @State private var showingAlert = false
            
            @State var registerErrorMessage = ""
            @State private var statusMessage = ""
            
            @State private var email1 = ""
            @State private var password1 = ""
            @State private var passwordConfirmation1 = ""
            @State private var firstName1 = ""
            @State private var lastName1 = ""
            
            @State var textFields: [String] = ["Last name","First name","Email"]
            @State var secureFields: [String] = ["Password", "Password confirmation"]
            
            @State var testObserver: AnyCancellable?

            var dataIsValid: Bool {
                return !registerProcessing && !email1.isEmpty && !password1.isEmpty && !passwordConfirmation1.isEmpty && password1 == passwordConfirmation1
            }
            var buttonColor: Color {
                return dataIsValid ? Color.theme.lighBlue : .gray
            }
        }
    \end{lstlisting}   
\end{minipage}

\hspace{-10mm}
\begin{minipage}{\textwidth}
    \linespread{0.8}\selectfont
    \begin{lstlisting}[language=swift]
        
        }
            var body: some View {
                NavigationView {
                    ZStack{
                        SetBackgroundImage(image: "Background3")
                        
                        VStack(alignment: .center){
                            HeaderView(title: "Registration", navigationPage: .welcomePage, arrowVisibility: true)
                            
                            VStack{
                                TextFieldView(text: $lastName1, title: "Last name")
                                TextFieldView(text: $firstName1, title: "First name")
                                TextFieldView(text: $email1, title: "Email")    
                                SecureFieldView(text: $password1, title: "Password")
                                SecureFieldView(text: $passwordConfirmation1, title: "Password confirmation")
                            }
                            .padding(.top,10)
                            VStack(spacing: 15) {
                                Button(action: {
                                    showingAlert = true
                                    handleAction()
                                }) {
                                    PrimaryButton(title: "Registration", fontColor: Color.theme.white, backgroundColor: buttonColor)
                                }
                                .disabled(!dataIsValid)
                                if registerProcessing {
                                    ProgressView()
                                }
                                
                                if !registerErrorMessage.isEmpty {
                                            Text("Failed creating account: \(registerErrorMessage)")
                                                .foregroundColor(.red)
                                }
                            }
                            HStack{
                                Text("Already have an account?")
                                    .font(.custom(.light, size: 14))
                                    .foregroundColor(Color.theme.white.opacity(0.7))
                                    .tracking(0.7)
                                
                                Button(action: {
                                    withAnimation {
                                        viewRouter.currentPage = .logInPage
                                    }
                                }) {
                                    Text("Login")
                                        .font(.custom(.bold, size: 16))
                                        .foregroundColor(Color.theme.white)
                                        .multilineTextAlignment(.leading)
                                }
                                .opacity(0.9)
                            }
                            .padding(.vertical)
                            VStack(spacing: 15){
                                Button(action: {
                                    registerSocialMedia()
                                }){
                                    SocialLoginButton(image: "GoogleLogo", text: "Sign in with Google")
                                }
                                SocialLoginButton(image: "AppleLogo", text: "Sing in with AppleID")
                            }
                            HStack{
                                Image("Logo2")
                                    .padding(.vertical,30)
                                    .frame(maxWidth: .infinity,alignment: .center)
                                    .shadow(color: Color.theme.darkGray.opacity(0.25), radius: 5, x: 0, y: 4)
                            }
                        }
                        .frame(width: UIScreen.main.bounds.width - 52)
                    }
                }
                .padding(0.0)
                .navigationBarHidden(false)
                .onAppear {
                    self.testObserver = self.email1.publisher(for: \.self).sink { newValue in
                        print("email1 changed, newValue: \(newValue)")
                    }
                }
            }
            
            
        }
    \end{lstlisting}   
\end{minipage}

\hspace{-10mm}
\begin{minipage}{\textwidth}
    \linespread{0.8}\selectfont
    \begin{lstlisting}[language=swift]
        private func handleAction() {
                registerUser()
            }
        private func registerUser(){
                registerProcessing = true
                Auth.auth().createUser(withEmail: email1, password: password1) { authResult, error in
                        guard error == nil else {
                            registerErrorMessage = error!.localizedDescription
                            registerProcessing = false
                            return
                        }     
                        switch authResult {
                        case .none:
                            print("Could not create account.")
                            registerProcessing = false
                        case .some(_):
                            print("User created")
                            registerProcessing = false
                            viewRouter.currentPage = .homePage
                        }
                    print("Successfully created user: \(authResult?.user.uid ?? "")")
                    self.viewModel.addUserData(firstName: self.firstName1, lastName: self.lastName1, email: self.email1, password: self.password1)
                    }
            }      
        private func registerSocialMedia(){
                        guard let clientID = FirebaseApp.app()?.options.clientID else { return }
                        let config = GIDConfiguration(clientID: clientID)
                        GIDSignIn.sharedInstance.signIn(with: config, presenting: getViewController()) { [self] user, err in
                            if let error = err {
                                print(error.localizedDescription)
                                return
                            }
                            guard
                                let auth = user?.authentication,
                                let userToken = auth.idToken
                            else {
                                return
                            }
                            let credential = AuthProvider.credential(withIDToken: userToken,
                            accessToken: auth.accessToken)
                        }
                    }
        Auth.auth().signIn(with: credential) { result, err in
                                switch result {
                                case .none:
                                    print("Could not create account.")
                                    registerProcessing = false
                                case .some(_):
                                    print("User created")
                                    registerProcessing = false
                                    viewRouter.currentPage = .homePage
                                }
                                self.viewModel.addUserData(firstName: (result?.user.firstname)! , lastName: (result?.user.lastname)!, email: (result?.user.email)!, password: result?.user.password)
                                
                            }
        func storeUserInformation() {
                        guard let uid = Auth.auth().currentUser?.uid else {return}
                        let userData = ["email": self.email1, "firstName": self.firstName1, "lastName": self.lastName1, "imageUrl": "imageurl", "id": uid]
                        Firestore.firestore().collection("users").document(uid).setData(userData) { err in
                            if let err = err {
                                return
                            }
                            print("Success")
                        }
                        
                    }
        struct RegisterView_Previews: PreviewProvider {
                    static var previews: some View {
                        RegisterView()
                    }
                }
    \end{lstlisting}   
\end{minipage}

\hspace{-10mm}
\begin{minipage}{\textwidth}
    \linespread{0.8}\selectfont
    \begin{lstlisting}[language=swift]
extension View {
            func getRect() ->CGRect{
                return UIScreen.main.bounds
            }
            func getViewController() -> UIViewController{
                guard let screen = UIApplication.shared.connectedScenes.first as? UIWindowScene else{
                    return .init()
                }
                guard let root = screen.windows.first?.rootViewController else{
                    return .init()
                }       
                return root
            }
        }
    \end{lstlisting}   
\end{minipage}


\subsection*{HomeView nézet}
\addcontentsline{toc}{subsection}{HomeView nézet}

\begin{spacing}{2}
\end{spacing}
\begin{minipage}{\textwidth}
    \linespread{0.8}\selectfont
    \begin{lstlisting}[language=swift]
        struct HomeView: View {
            
            @ObservedObject var viewModel = ViewModel()
            
            @State var searchText = ""
            @State var maxCount = 20
             
            var body: some View {
                ZStack{
                    SetBackgroundImage(image: "Background3")
                    
                    VStack(){
                        HStack{
                            Image("Logo2")
                                .padding(.top, 60)
                                .frame(maxWidth: .infinity,alignment: .center)
                                .shadow(color: Color.theme.darkGray.opacity(0.25), radius: 5, x: 0, y: 4)
                        }
                        VStack(alignment: .leading, spacing: 15){
                            HStack{
                                Text("Hi , Robi!")
                                    .font(.custom(.regular, size: 22))
                                    .foregroundColor(Color.theme.white)
                                
                                Spacer()
                                
                                Image("noProfile")
                                    .resizable()
                                    .clipShape(Circle())
                                    .overlay(Circle().stroke((Color.theme.white), lineWidth: 2))
                                    .frame(width: 40, height: 40)
                            }
                            Text("Get stronger today!")
                                .font(.custom(.light, size: 20))
                                .foregroundColor(Color.theme.white)
                        }
                        HStack{
                            Image(systemName: "magnifyingglass")
                                .foregroundColor(
                                    viewModel.searchText.isEmpty ? Color.theme.darkGray : Color.theme.lighBlue
                                )     
                            TextField("Search by training plan...", text: $viewModel.searchText)
                                .foregroundColor(Color.theme.darkGray)
                                .onChange(of: viewModel.searchText) { text in
                                                viewModel.filterContent()
                                    var checked = viewModel.searchText.trimmingCharacters(in: CharacterSet.alphanumerics.inverted)
                                checked = String(checked.prefix(maxCount))
                                    viewModel.searchText = checked
                                                }
                                
                }   
            }
        }  
    \end{lstlisting}   
\end{minipage}


\hspace{-10mm}
\begin{minipage}{\textwidth}
    \linespread{0.8}\selectfont
    \begin{lstlisting}[language=swift]
                            .overlay(
                                    Image(systemName: "xmark.circle.fill")
                                        .padding()
                                        .offset(x: 10)
                                        .opacity(viewModel.searchText.isEmpty ? 0.0 : 1.0)
                                        .foregroundColor(Color.theme.lighBlue)
                                        .onTapGesture {
                                            viewModel.searchText = ""
                                        }
                                    ,alignment: .trailing
                                )
                                .onChange(of: viewModel.searchText) { text in
                                    viewModel.filterContent()
                                }           
                        }
                        .font(.headline)
                        .padding()
                        .background(RoundedRectangle(cornerRadius: 15)
                            .fill(Color.theme.white))
                        .shadow(color: Color.theme.darkGray.opacity(0.25), radius: 5, x: 0, y: 4)  
                        HStack{
                            Text("My collection:")
                                .font(.custom(.regular, size: 18))
                                .foregroundColor(Color.theme.white)
                                .padding(.top)
                            Spacer()
                        }
                    }
                    .frame(width: UIScreen.main.bounds.width - 52)
        ScrollView(.vertical, showsIndicators: false){
            if $viewModel.filteredPlan.count > 0 {
                ForEach(viewModel.filteredPlan) { traningPlan in
                    
                    TrainingPlanCardView(traningPlan: traningPlan)
                    
                }
                .frame(maxWidth: .infinity)
            } else {
                HStack(alignment: .center){
                    Spacer()
                    Text("No training plan found")
                        .font(.custom(.regular, size: 20))
                        .foregroundColor(Color.theme.white)
                    Spacer()
                }
                .padding(.top,200)
            }
        }  
        struct HomeView_Previews: PreviewProvider {
            static var previews: some View {
                HomeView()
                    .environmentObject(ViewModel())
            
            }
        }
    \end{lstlisting}   
\end{minipage}

\subsection*{TabBarView nézet}
\addcontentsline{toc}{subsection}{TabBarView nézet}

\begin{spacing}{2}
\end{spacing}
\begin{minipage}{\textwidth}
    \linespread{0.8}\selectfont
    \begin{lstlisting}[language=swift]
        struct TabBarView: View {
            @State var selectedTab = "house"
            
            init(){
                
                UITabBar.appearance().isHidden = true
            }
            var body: some View {
                ZStack(alignment: Alignment(horizontal: .center, vertical: .bottom)){
                    
                    TabView(selection: $selectedTab){
                        HomeView()
                            .tag("house")
                        ScannerView()
                            .tag("qrcode.viewfinder")
                        ProfileView()
                            .tag("person")
                    }
                }
            }
        }

    \end{lstlisting}   
\end{minipage}

\hspace{-10mm}
\begin{minipage}{\textwidth}
    \linespread{0.8}\selectfont
    \begin{lstlisting}[language=swift]
        HStack(spacing: 0){
            ForEach(tabs, id:\.self){ image in
                Button {
                    withAnimation(.spring()){
                        selectedTab = image
                    }
                } label: {
                        VStack(alignment: .center, spacing: 4){
                            Image(systemName: image)
                                .resizable()
                                .renderingMode(.template)
                                .aspectRatio(contentMode:.fit)
                                .frame(width: 32, height: 32)
                                .foregroundColor(selectedTab == image ?  Color.theme.lighBlue : Color.theme.white)
                        }
                }
            }
        }
                    {
                         if image != tabs.last{
                            Spacer(minLength: 0)
                        }
                    }
                    .padding(.horizontal,50)
                    .padding(.vertical, 20)
                    .shadow(color: Color.theme.darkGray.opacity(0.5), radius:4, x:0, y: -1)
                    .background(Color.theme.darkGray)
                    .cornerRadius(30)
                    .ignoresSafeArea(.all, edges: .bottom)
            
        
        var tabs = ["house", "qrcode.viewfinder", "person"]

        struct TabBarView_Previews: PreviewProvider {
            static var previews: some View {
                TabBarView()
            }
        }

    \end{lstlisting}   
\end{minipage}

\subsection*{TraningPlanView nézet}
\addcontentsline{toc}{subsection}{TraningPlanView nézet}

\begin{spacing}{2}
\end{spacing}
\begin{minipage}{\textwidth}
    \linespread{0.8}\selectfont
    \begin{lstlisting}[language=swift]
        struct TraningPlanView: View {
            @ObservedObject var viewModel = ViewModel()
            var traningPlan: TraningPlan
            var body: some View {
                VStack{
                    ZStack(alignment: .top){
                        Image("Background5")
                            .frame(height: 325)
                            
                        VStack(alignment: .leading){
                            HeaderWithLogoView(qrVisibility: false,arrowVisibility:true, navigationPage: .homePage)
                                .padding(.top)
                            
                            Text(traningPlan.name)
                                .foregroundColor(Color.theme.white)
                                .font(.custom(.light, size: 30))
                                .padding(.top, 130)
                            HStack(spacing: 170){
                                HStack{
                                    Image(systemName: "stopwatch")
                                        .frame(width: 18,height: 18)
                                    Text("\(traningPlan.duration) min")
                                        .font(.custom(.regular, size: 15))
                                }
                            }
                            .foregroundColor(Color.theme.white)
                            
                            VStack{
                                Text("Exercises")
                                    .foregroundColor(Color.theme.white)
                                    .font(.custom(.medium, size: 14))
                                    .frame(maxWidth: .infinity, alignment: .center)
                                    .padding()
                                    .background(Color.theme.lighBlue)
                                    .cornerRadius(20)
                                    .padding(.top)
                            }
                            
    \end{lstlisting}   
\end{minipage}

\hspace{-10mm}
\begin{minipage}{\textwidth}
    \linespread{0.8}\selectfont
    \begin{lstlisting}[language=swift]
        ScrollView{
            if $viewModel.exercise.count > 0 {
                    ForEach(viewModel.exercise){ exercise in
                        ExerciseCardView(exercise: exercise)
                }
                .padding(.vertical, 5)
            }
        }
        .padding(.vertical, 15)
                        }
                        .frame(width: UIScreen.main.bounds.width - 52)
                    }
                }
                .background(Color.theme.darkGray)
                .ignoresSafeArea()
            }
        }
        struct TraningPlanView_Previews: PreviewProvider {
            static var previews: some View {
                TraningPlanView()
            }
        }  
    \end{lstlisting}   
\end{minipage}

\subsection*{ExerciseDetailView nézet}
\addcontentsline{toc}{subsection}{ExerciseDetailView nézet}

\begin{spacing}{2}
\end{spacing}
\begin{minipage}{\textwidth}
    \linespread{0.8}\selectfont
    \begin{lstlisting}[language=swift]
        struct ExerciseDetailView: View {
            @State var showMapView:Bool = false
            var exercise: Exercise
            
            var body: some View {
                VStack{
                    HStack(alignment: .top){
                        VStack(alignment: .leading){
                            VStack(alignment: .center, spacing: 40){
                                HeaderWithLogoView(qrVisibility: true,arrowVisibility:true, navigationPage: .traningPage)
                                    .padding(.top)
                                Image(exercise.imageUrl)
                                    .resizable()
                                    .aspectRatio(contentMode: .fit)
                                    .frame(height: 300)
                                    .cornerRadius(10)
                            }
                            VStack(alignment: .leading, spacing: 10){
                                HStack{
                                    Text(exercise.name)
                                        .font(.custom(.medium, size: 22))
                                    Spacer()
                                    Button {
                                        showMapView.toggle()
                                    } label: {
                                        ZStack {
                                            Circle()
                                                .fill(Color.theme.lighBlue)
                                                .frame(width: 35, height: 35)
                                            
                                            Text(exercise.machineNumber).font(.custom(.medium, size: 20))
                                                .foregroundColor(Color.theme.white)
                                        }
                                    }
                                    .sheet(isPresented: $showMapView, content: {
                                        MapView()
                                    })
                            }
                                ScrollView(.vertical, showsIndicators: false){
                                    VStack(alignment: .leading, spacing: 10){
                                        Text("Initial situation")
                                            .font(.custom(.regular, size: 18))
                                        Text(exercise.description)
                                            .font(.custom(.light, size: 14))
                                        Text("The movement")
                                            .font(.custom(.regular, size: 18))
                                        Text(exercise.moment)
                                            .font(.custom(.light, size: 14))
                                        Text("Workout tip")
                                            .font(.custom(.regular, size: 18))
                                        Text(exercise.tip)
                                            .font(.custom(.light, size: 14))
                                    }
                                }
                            }
    \end{lstlisting}   
\end{minipage}

\hspace{-10mm}
\begin{minipage}{\textwidth}
    \linespread{0.8}\selectfont
    \begin{lstlisting}[language=swift]
        .padding(.bottom)
                            .foregroundColor(Color.theme.white)
                            .padding(.top, 40)
                        }
                        .frame(width: UIScreen.main.bounds.width - 52)
                    }
                    .edgesIgnoringSafeArea(.all)
                    .frame(maxWidth: .infinity)
                    .background(Color.theme.darkGray)
                }
            }
        }
                
        struct ExerciseDetailView_Previews: PreviewProvider {
            static var previews: some View {
                ExerciseDetailView()
                    .preferredColorScheme(.dark)
            }
        }
    \end{lstlisting}   
\end{minipage}

\subsection*{ProfileView nézet}
\addcontentsline{toc}{subsection}{ProfileView nézet}

\begin{spacing}{2}
\end{spacing}
\begin{minipage}{\textwidth}
    \linespread{0.8}\selectfont
    \begin{lstlisting}[language=swift]

        struct ProfileView: View {
            
            @ObservedObject var viewModel = ViewModel()

            @State var email = ""
            @State var firstName = ""
            @State var lastName = ""
            @State var password = ""
            @State var passwordConfirmation = ""
            
            @State var saveProcessing = false
            @State var logOutProcessing = false
            @State var showLogoutOption = false
            
            var dataIsValid: Bool {
                return !email.isEmpty && !password.isEmpty && !firstName.isEmpty && !lastName.isEmpty && !passwordConfirmation.isEmpty && password == passwordConfirmation
            }
            var buttonColor: Color {
                return dataIsValid ? Color.theme.lighBlue : .gray
            }
            @EnvironmentObject var viewRouter: ViewRouter
            var body: some View {
                ZStack{
                    Image("Background2")
                        
                    VStack{
                        HeaderView(title: "Edit profile", navigationPage: .homePage, arrowVisibility: false)
                            .padding(.top,40)
                            .padding(.bottom,30)
                        Spacer()

                        ZStack{
                            Image("noProfile")
                                .resizable()
                                .aspectRatio(contentMode: .fill)
                                .frame(width: 118, height: 118)
                                .scaledToFit()
                                .cornerRadius(4)
                                .clipShape(RoundedRectangle(cornerRadius: 4))
                                RoundedRectangle(cornerRadius: 4)
                                    .strokeBorder(Color.theme.white, lineWidth: 4)
                                    .compositingGroup()
                                    .frame(width: 118, height: 118)
                                    .shadow(color: Color.theme.darkGray,radius: -4, x:0, y:4)
                        }
                        
                        VStack{
                            ScrollView{
                                ForEach(viewModel.user){ data in
                                        TextFieldView(text: $email, title: "Email")
                                        TextFieldView(text: $lastName, title: "Last name")
                                        TextFieldView(text: $firstName, title: "First name")
                                        SecureFieldView(text: $password, title: "Password")
                                        SecureFieldView(text: $passwordConfirmation, title: "Password confirmation")
                                            .padding(.bottom, 40)
                                }
                            }
    \end{lstlisting}   
\end{minipage}

\hspace{-10mm}
\begin{minipage}{\textwidth}
    \linespread{0.8}\selectfont
    \begin{lstlisting}[language=swift]
                        Button {
                                viewModel.updateUser(email: self.email, password: self.password, firstName: self.firstName, lastName: self.lastName)
                            } label: {
                                PrimaryButton(title: "Save", fontColor: Color.theme.white, backgroundColor: buttonColor)
                            }
                                    .disabled(!dataIsValid)
                            
                                if saveProcessing {
                                    ProgressView()
                                }
                                if !viewModel.saveErrorMessage.isEmpty {
                                    Text("Failed: \(viewModel.saveErrorMessage)")
                                                    .foregroundColor(.red)
                                }
                            Button {
                                showLogoutOption.toggle()
                            } label: {
                                PrimaryButton(title: "Log out", fontColor: Color.theme.white, backgroundColor: Color.theme.red)
                                    .padding(.bottom, 140)
                            }
                            .actionSheet(isPresented: $showLogoutOption){
                                .init(title: Text("Sure you want to log out?"),buttons: [
                                    .destructive(Text("Log Out"),action: {
                                        withAnimation{
                                            logOutUser()
                                        }
                                    }),
                                    .cancel()
                                ])
                            }
                        }
                        
                    }
                    .frame(width: UIScreen.main.bounds.width - 52)
                }
                .background(Color.theme.gray)
                .ignoresSafeArea()
                .frame(maxWidth: .infinity)
            }
            
            func logOutUser() {
                logOutProcessing = true
                let firebaseAuth = Auth.auth()
                do {
                try firebaseAuth.signOut()
                } catch let logOutError as NSError {
                print("Logout error: %@", logOutError)
                    logOutProcessing = false
                }
                withAnimation{
                    viewRouter.currentPage = .logInPage
                }
            
            }
        }

        struct ProfileView_Previews: PreviewProvider {
            static var previews: some View {
                TabBarView()
            }
        }
    \end{lstlisting}   
\end{minipage}

\subsection*{MapView nézet}
\addcontentsline{toc}{subsection}{MapView nézet}

\begin{spacing}{2}
\end{spacing}
\begin{minipage}{\textwidth}
    \linespread{0.8}\selectfont
    \begin{lstlisting}[language=swift]
        struct MapView: View {
            
            @State var currentScale: CGFloat = 0
            @State var finalScale: CGFloat = 1
            
            var body: some View {
                VStack{
                    HeaderWithLogoView(qrVisibility: true,arrowVisibility: false, navigationPage: .homePage)
                    Spacer()
                    Image("gymMap")
                        .resizable()
                        .aspectRatio(contentMode: .fill)
                        .frame(width: .infinity, height: 200)
                        .scaleEffect(currentScale + finalScale)
                        .gesture(
                        MagnificationGesture()
                            .onChanged{ newScale in
                                currentScale = newScale
                            }
                            .onEnded{ scale in
                                finalScale = scale
                                currentScale = 0
                            }
                        )
                    Spacer()
                }
                .frame(width: UIScreen.main.bounds.width - 52)
                .scaledToFill()
                .edgesIgnoringSafeArea(.all)
                .frame(maxWidth: .infinity)
                .background(Color.theme.darkGray)
            }
        }

        struct MapView_Previews: PreviewProvider {
            static var previews: some View {
                MapView()
            }
        }
    \end{lstlisting}   
\end{minipage}

\subsection*{QrCodeScannerView nézet}
\addcontentsline{toc}{subsection}{QrCodeScannerView nézet}

\begin{spacing}{2}
\end{spacing}
\begin{minipage}{\textwidth}
    \linespread{0.8}\selectfont
    \begin{lstlisting}[language=swift]
        struct QrCodeScannerView: UIViewRepresentable {
            var supportedBarcodeTypes: [AVMetadataObject.ObjectType] = [.qr]
            typealias UIViewType = CameraPreview
            
            private let session = AVCaptureSession()
            private let delegate = CameraDelegate()
            private let metadataOutput = AVCaptureMetadataOutput()
            
            func torchLight(isOn: Bool) -> QrCodeScannerView {
                if let backCamera = AVCaptureDevice.default(for: AVMediaType.video) {
                    if backCamera.hasTorch {
                        try? backCamera.lockForConfiguration()
                        if isOn {
                            backCamera.torchMode = .on
                        } else {
                            backCamera.torchMode = .off
                        }
                        backCamera.unlockForConfiguration()
                    }
                }
                return self
            }
            func interval(delay: Double) -> QrCodeScannerView {
                delegate.scanInterval = delay
                return self
            }
            func found(r: @escaping (String) -> Void) -> QrCodeScannerView {
                delegate.onResult = r
                return self
            }
    \end{lstlisting}   
\end{minipage}

\hspace{-10mm}
\begin{minipage}{\textwidth}
    \linespread{0.8}\selectfont
    \begin{lstlisting}[language=swift]
    }
        func makeUIView(context: UIViewRepresentableContext<QrCodeScannerView>) = QrCodeScannerView.UIViewType {
        let cameraView = CameraPreview(session: session)
        
        if targetEnvironment(simulator)
        cameraView.createSimulatorView(delegate: self.delegate)
        else
        checkCameraAuthorizationStatus(cameraView)
        endif
        
        return cameraView
        static func dismantleUIView(_ uiView: CameraPreview, coordinator: ()) {
        uiView.session.stopRunning()
        }
    
        private func checkCameraAuthorizationStatus(_ uiView: CameraPreview) {
            let cameraAuthorizationStatus = AVCaptureDevice.authorizationStatus(for: .video)
            if cameraAuthorizationStatus == .authorized {
                setupCamera(uiView)
            } else {
                AVCaptureDevice.requestAccess(for: .video) { granted in
                    DispatchQueue.main.sync {
                        if granted {
                            self.setupCamera(uiView)
                        }
                    }
                }
            }
        }
        func updateUIView(_ uiView: CameraPreview, context: UIViewRepresentableContext<QrCodeScannerView>) {
            uiView.setContentHuggingPriority(.defaultHigh, for: .vertical)
            uiView.setContentHuggingPriority(.defaultLow, for: .horizontal)
        }
    }
    \end{lstlisting}   
\end{minipage}

\subsection*{ScannerView nézet}
\addcontentsline{toc}{subsection}{ScannerView nézet}

\begin{spacing}{2}
\end{spacing}
\begin{minipage}{\textwidth}
    \linespread{0.8}\selectfont
    \begin{lstlisting}[language=swift]
        struct ScannerView: View {
    
            @ObservedObject var viewModel = ScannerViewModel()
            
            var body: some View {
                ZStack{
                    QrCodeScannerView()
                        .found(r: self.viewModel.onFoundQrCode)
                        .torchLight(isOn: self.viewModel.torchIsOn)
                        .interval(delay: self.viewModel.scanIterval)
                    Image("qrcode")
                    
                    VStack{
                        VStack{
                            Text("Keep scanning for QR-codes")
                                .font(.custom(.medium, size: 14))
                                .foregroundColor(Color.theme.white)
                                .padding(.vertical, 17)
                                .frame(maxWidth: .infinity, alignment: .center)
                                .background(Color.theme.lighBlue)
                                .cornerRadius(5)
                                .shadow(color: Color.theme.darkGray.opacity(0.25), radius: 5, x: 0, y: 4)
                                .padding()
                    
                            Text(self.viewModel.lastQrCode)
                                .foregroundColor(Color.theme.white)
                                .bold()
                                .lineLimit(5)
                                .padding()
                        }
                        .padding(.vertical, 10)
    \end{lstlisting}   
\end{minipage}


\begin{spacing}{2}
\end{spacing}
\hspace*{-10mm}
\begin{minipage}{\textwidth}
    \linespread{0.8}\selectfont
    \begin{lstlisting}[language=swift]
            
        HStack{
            Button(action: {
                self.viewModel.torchIsOn.toggle()
            }, label: {
                Image(systemName: self.viewModel.torchIsOn ? "flashlight.on.fill" : "flashlight.off.fill")
                    .font(.system(size: 36))
                    .foregroundColor(self.viewModel.torchIsOn ? Color.theme.lighBlue : Color.theme.white)
            })
            .padding(.bottom, 50)
        }
        
        struct ScannerView_Previews: PreviewProvider {
            static var previews: some View {
                ScannerView()
            }
        }
    \end{lstlisting}   
\end{minipage}

\subsection*{MotherView nézet}
\addcontentsline{toc}{subsection}{MotherView nézet}

\begin{spacing}{2}
\end{spacing}
\begin{minipage}{\textwidth}
    \linespread{0.8}\selectfont
    \begin{lstlisting}[language=swift]
        struct MotherView: View {
            @EnvironmentObject var viewRouter: ViewRouter
            
            var body: some View {
                switch viewRouter.currentPage {
                case .welcomePage:
                    WelcomeView()
                case .registerPage:
                    RegisterView()
                case .logInPage:
                    LoginView()
                case .homePage:
                    TabBarView()
                case .traningPage:
                    TraningPlanView()
                case .exercisePage:
                    ExerciseDetailView()
                case .qrPage:
                    ScannerView()
                }
            }
        }

        struct MotherView_Previews: PreviewProvider {
            static var previews: some View {
                MotherView().environmentObject(ViewRouter())
            }
        }
    \end{lstlisting}   
\end{minipage}

\section*{Komponensek} 
\addcontentsline{toc}{section}{Komponensek}

\subsection*{ExerciseCardView komponens}
\addcontentsline{toc}{subsection}{ExerciseCardView komponens}

\begin{spacing}{2}
\end{spacing}
\begin{minipage}{\textwidth}
    \linespread{0.8}\selectfont
    \begin{lstlisting}[language=swift]

        struct ExerciseCardView: View {
            @ObservedObject var viewModel = ViewModel()
            @EnvironmentObject var viewRouter: ViewRouter
            @State private var showingTraningPlan = false
            var exercise: Exercise
            
            var body: some View {
                Button(action: {
                    withAnimation {
                        viewRouter.currentPage = .exercisePage
                    }
                }){
                    HStack(alignment: .center, spacing: 15){
                        Image(exercise.imageUrl)
                            .resizable()
                            .aspectRatio(contentMode: .fill)
                            .frame(width: 51, height: 51)
                        
                        VStack(alignment: .leading, spacing: 8){
                            Text(exercise.name)
                                .font(.custom(.medium, size: 13))
                            Text(exercise.setOf)
                                .font(.custom(.regular, size: 12))
                        }
                        
                        Image(systemName: "chevron.right")
                            .frame(width: 24, height: 24)
                            .background(Color.theme.lighBlue)
                            .cornerRadius(50)
                    }
                    .padding()
                    .foregroundColor(Color.theme.white)
                    .frame(maxWidth: 350, alignment: .center)
                    .background(Color.theme.gray.opacity(0.3))
                    .cornerRadius(15)
                    .shadow(color: Color.theme.darkGray.opacity(0.25), radius: 5, x: 0, y: 4)
                    .overlay(RoundedRectangle(cornerRadius: 15)
                                .stroke(Color.theme.white, lineWidth: 2))
                }
            }
        }

        struct ExerciseCardView_Previews: PreviewProvider {
            static var previews: some View {
                Group{
                    ExerciseCardView(exercise: viewModel.exercise)
                        .previewLayout(.sizeThatFits)
                        .preferredColorScheme(.dark)
                    ExerciseCardView()
                        .previewLayout(.sizeThatFits)
                        .preferredColorScheme(.light)
                        
                }
                TraningPlanView()
                
            }
        }

    \end{lstlisting}   
\end{minipage}

\subsection*{HeaderView komponens}
\addcontentsline{toc}{subsection}{HeaderView komponens}

\begin{spacing}{2}
\end{spacing}
\begin{minipage}{\textwidth}
    \linespread{0.8}\selectfont
    \begin{lstlisting}[language=swift]
        struct HeaderView: View {
            
            @State var title: String
            @State var navigationPage: Page
            @State var arrowVisibility:Bool
            @EnvironmentObject var viewRouter: ViewRouter
            
            var body: some View {
                HStack{
                    if arrowVisibility {
                        Button(action:{
                            viewRouter.currentPage = navigationPage
                        }) {
                            Image(systemName: "arrow.left")
                                .foregroundColor(Color.theme.white)
                                .font(.system(size: 30))
                        }
                    }
                    
                    Spacer()
                }
                .overlay(content: {
                    Text(title)
                        .foregroundColor(.white)
                        .font(.custom(.medium, size: 24))
                })
                .padding(.top, 50)
            }
        }
        
        struct HeaderView_Previews: PreviewProvider {
            static var previews: some View {
                HeaderView(title: "Login", navigationPage: .welcomePage, arrowVisibility: true)
                    .previewLayout(.sizeThatFits)
                    .preferredColorScheme(.dark)
            }
        }
        
    \end{lstlisting}   
\end{minipage}

\subsection*{HeaderWithLogoView komponens}
\addcontentsline{toc}{subsection}{HeaderWithLogoView komponens}

\begin{spacing}{2}
\end{spacing}
\begin{minipage}{\textwidth}
    \linespread{0.8}\selectfont
    \begin{lstlisting}[language=swift]
        struct HeaderWithLogoView: View {
            
            @State var qrVisibility: Bool
            @State var arrowVisibility: Bool
            
            @ObservedObject var viewModel = ViewModel()
            @EnvironmentObject var viewRouter: ViewRouter
            
            @State var navigationPage: Page
            @State var showMapView:Bool = false
            
            var body: some View {
                    HStack{
                        if self.arrowVisibility{
                            Button {
                                withAnimation {
                                    viewRouter.currentPage = navigationPage
                                }
                            } label: {
                                Image(systemName: "arrow.left")
                                    .foregroundColor(Color.theme.white)
                                    .font(.system(size: 30))
                            }
                        }
                        
                    }}}
    \end{lstlisting}   
\end{minipage}



\begin{spacing}{2}
\end{spacing}
\hspace*{-10mm}
\begin{minipage}{\textwidth}
    \linespread{0.8}\selectfont
    \begin{lstlisting}[language=swift]
                if self.qrVisibility{
                            Button {
                                showMapView.toggle()
                            } label: {
                                Image(systemName: "qrcode.viewfinder")
                                    .foregroundColor(Color.theme.lighBlue)
                                    .font(.system(size: 30))
                            }
                            .sheet(isPresented: $showMapView, content: {
                                ScannerView()
                            })
                        }
                    }
                    .overlay(content: {
                            Image("Logo2")
                                .foregroundColor(.white)
                                .font(.custom(.medium, size: 24))
                    })
                    .padding(.top, 50)
            }
        }

        struct HeaderWithLogoView_Previews: PreviewProvider {
            static var previews: some View {
                HeaderWithLogoView(qrVisibility: true, arrowVisibility: true, navigationPage: .logInPage)
                    .previewLayout(.sizeThatFits)
                    .preferredColorScheme(.dark)
            }
        }

    \end{lstlisting}   
\end{minipage}


\subsection*{PrimaryButton komponens}
\addcontentsline{toc}{subsection}{PrimaryButton komponens}

\begin{spacing}{2}
\end{spacing}
\begin{minipage}{\textwidth}
    \linespread{0.8}\selectfont
    \begin{lstlisting}[language=swift]
        struct PrimaryButton: View {
    
            var title: String
            var fontColor: Color
            var backgroundColor: Color
            @StateObject var viewRouter = ViewRouter()
            
            var body: some View {
                    Text(title)
                        .font(.custom(.medium, size: 20))
                        .foregroundColor(fontColor)
                        .padding(.vertical, 17)
                        .frame(maxWidth: .infinity, alignment: .center)
                        .background(backgroundColor)
                        .cornerRadius(5)
                        .shadow(color: Color.theme.darkGray.opacity(0.25), radius: 5, x: 0, y: 4)
            }
        }
        
        struct PrimaryButtonView_Previews: PreviewProvider {
            static var previews: some View {
                PrimaryButton(title: "Next", fontColor: Color.theme.white, backgroundColor: Color.theme.darkGray)
                    .padding()
                    .previewLayout(.sizeThatFits)
                
            }
        }
    \end{lstlisting}   
\end{minipage} 


\subsection*{SearchBarView komponens}
\addcontentsline{toc}{subsection}{SearchBarView komponens}

\begin{spacing}{2}
\end{spacing}
\begin{minipage}{\textwidth}
    \linespread{0.8}\selectfont
    \begin{lstlisting}[language=swift]
        struct SearchBarView: View {
            
            @ObservedObject var viewModel = ViewModel()
            
            var body: some View {
                HStack{
                    Image(systemName: "magnifyingglass")
                        .foregroundColor(
                            viewModel.searchText.isEmpty ? Color.theme.darkGray : Color.theme.lighBlue
                        )
                    
                    TextField("Search by training plan...", text: $viewModel.searchText)
                        .foregroundColor(Color.theme.darkGray)
                        .onChange(of: viewModel.searchText) { text in
                                            viewModel.filterContent()
                                        }
                        .overlay(
                            Image(systemName: "xmark.circle.fill")
                                .padding()
                                .offset(x: 10)
                                .opacity(viewModel.searchText.isEmpty ? 0.0 : 1.0)
                                .foregroundColor(Color.theme.lighBlue)
                                .onTapGesture {
                                    viewModel.searchText = ""
                                }
                            ,alignment: .trailing
                        )
                }
                .font(.headline)
                .padding()
                .background(RoundedRectangle(cornerRadius: 15)
                    .fill(Color.theme.white))
                .shadow(color: Color.theme.darkGray.opacity(0.25), radius: 5, x: 0, y: 4)
            }
        }

        struct SearchBarView_Previews: PreviewProvider {
            static var previews: some View {
                Group{
                    SearchBarView()
                        .previewLayout(.sizeThatFits)
                        .preferredColorScheme(.light)
                    
                    SearchBarView()
                        .previewLayout(.sizeThatFits)
                        .preferredColorScheme(.dark)
                }
                
            }
        }
    \end{lstlisting}   
\end{minipage}

\subsection*{SecureFieldView komponens}
\addcontentsline{toc}{subsection}{SecureFieldView komponens}

\begin{spacing}{2}
\end{spacing}
\begin{minipage}{\textwidth}
    \linespread{0.8}\selectfont
    \begin{lstlisting}[language=swift]

        struct SecureFieldView: View {
            
            @Binding var text: String
            @State var title: String
            @State var visibility = false
            
            var body: some View {
                HStack{
                    
                    ZStack(alignment: .leading){
                        
                        if text.isEmpty{
                            Text(title)
                                .foregroundColor(Color.theme.white.opacity(0.5))
                                .font(.custom(.light, size: 20))
                        }
                        if self.visibility{
                            TextField("", text: self.$text)
                        }else{
                            SecureField("", text: self.$text)
                        }
                    }
    \end{lstlisting}   
\end{minipage}
\begin{spacing}{2}
\end{spacing}
\hspace{-10mm}
\begin{minipage}{\textwidth}
    \linespread{0.8}\selectfont
    \begin{lstlisting}[language=swift]
        Button(action: {
                                self.visibility.toggle()
                            })  {
                                Image(systemName: self.visibility ? "eye.slash" : "eye.fill")
                                    .foregroundColor(Color.theme.white)
                                    .padding(.trailing, 5)
                                    .font(.system(size: 16))
                            }
                        }
                        .overlay(Rectangle().frame(height: 2).padding(.top, 35))
                        .foregroundColor(.white)
                        .font(.custom(.light, size: 20))
                        .padding(.vertical, 20)
                    }
                }

                struct SecureFieldView_Previews: PreviewProvider {
                    @State static var test = "hello"
                    
                    static var previews: some View {        
                        SecureFieldView(text: $test, title: "Password")
                            .previewLayout(.sizeThatFits)
                            .preferredColorScheme(.dark)
                    }
        }
    \end{lstlisting}   
\end{minipage}

\subsection*{SecureFieldView komponens}
\addcontentsline{toc}{subsection}{SecureFieldView komponens}

\begin{spacing}{2}
\end{spacing}
\begin{minipage}{\textwidth}
    \linespread{0.8}\selectfont
    \begin{lstlisting}[language=swift]
        struct SocialLoginButton: View {
            var image: String
            var text: String
            
            var body: some View {
                    Label(text, image: image)
                        .font(.custom(.light, size: 16))
                        .foregroundColor(Color.theme.darkGray)
                        .padding(.vertical, 17)
                        .frame(maxWidth: .infinity, alignment: .center)
                        .background(Color.theme.white)
                        .cornerRadius(5)
                        .shadow(color: Color.theme.darkGray.opacity(0.25), radius: 5, x: 0, y: 4)
            }
        }
        
        struct SocialLoginButtonView_Previews: PreviewProvider {
            static var previews: some View {
                SocialLoginButton(image: "GoogleLogo", text: "Sign in with Google")
                    .padding()
                    .previewLayout(.sizeThatFits)
                
            }
        }
    \end{lstlisting}   
\end{minipage}

\subsection*{TextFieldView komponens}
\addcontentsline{toc}{subsection}{TextFieldView komponens}

\begin{spacing}{2}
\end{spacing}
\begin{minipage}{\textwidth}
    \linespread{0.8}\selectfont
    \begin{lstlisting}[language=swift]
        struct TextFieldView: View {
    
            @Binding var text: String
            @State var title: String
            
            var body: some View {
                ZStack(alignment: .leading){
                    if text.isEmpty{
                        Text(title)
                            .padding(.top,20)
                            .foregroundColor(Color.theme.white.opacity(0.5))
                            .font(.custom(.light, size: 20))
                            .padding(.vertical, 10)
                    }            
    \end{lstlisting}   
\end{minipage}

\begin{spacing}{2}
\end{spacing}
\hspace{-10mm}
\begin{minipage}{\textwidth}
    \linespread{0.8}\selectfont
    \begin{lstlisting}[language=swift]
        TextField("", text: $text)
                                .overlay(Rectangle().frame(height: 2).padding(.top, 35))
                                .padding(.top,20)
                                .font(.custom(.light, size: 20))
                                .foregroundColor(.white)
                                .padding(.vertical, 10)
                        }
                    }
                }
        struct TextFieldView_Previews: PreviewProvider {
            @State static var test: String = ""
            
            static var previews: some View {
                TextFieldView(text: $test, title: "Email")
                    .background(.black)
            }
        }
    \end{lstlisting}   
\end{minipage}

\subsection*{TrainingPlanCardView komponens}
\addcontentsline{toc}{subsection}{TrainingPlanCardView komponens}

\begin{spacing}{2}
\end{spacing}
\begin{minipage}{\textwidth}
    \linespread{0.8}\selectfont
    \begin{lstlisting}[language=swift]
        struct TrainingPlanCardView: View {

            @ObservedObject var viewModel = ViewModel()
            @EnvironmentObject var viewRouter: ViewRouter 
            @State private var showingTraningPlan = false
            var traningPlan: TraningPlan 
            var body: some View {
                    ZStack{
                        AsyncImage(url: URL(string:traningPlan.imageUrl))
                                    .frame(width: 340,height: 160)
                                    .cornerRadius(15)
        
                                HStack{
                                    Spacer()
                                    Button(action: {
                                        withAnimation {
                                            viewRouter.currentPage = .traningPage
                                        }
                                    }){
                                        HStack{
                                            Text("View at")
                                                .font(.custom(.medium, size: 30))
                                                .foregroundColor(Color.theme.white)
                                                .shadow(color: Color.theme.darkGray.opacity(0.25), radius: 5, x: 0, y: 4)
                                            ZStack{
                                                RoundedRectangle(cornerRadius: 50)
                                                    .fill(Color.theme.lighBlue)
                                                    .frame(width: 32, height: 32)
                                                Image(systemName: "chevron.right")
                                                    .foregroundColor(Color.theme.white)
                                                    .font(.system(size: 20))
                                            }
                                        }
                                        .padding(.trailing)
                                    }
                                }
                        
                                HStack{
                                    Image(traningPlan.imageUrl)
                                    Spacer()
                                }
                                
                                VStack(alignment: .leading, spacing: 10 ){
                                    Text(traningPlan.level)
                                        .font(.custom(.light, size: 14))
                                        .foregroundColor(Color.theme.lighBlue)
                                    Text(traningPlan.name)
                                        .font(.custom(.regular, size: 20))
                                        .foregroundColor(Color.theme.white)
                                    Spacer()
                                    HStack(){
                                        Image(systemName: "stopwatch")
                                            .foregroundColor(Color.theme.white)
                                            .font(.system(size: 14))
                                        Text("\(traningPlan.duration) min")
                                            .font(.custom(.light, size: 14))
                                            .foregroundColor(Color.theme.white)
                                    }
    \end{lstlisting}   
\end{minipage}

\begin{spacing}{2}
\end{spacing}
\hspace{-10mm}
\begin{minipage}{\textwidth}
    \linespread{0.8}\selectfont
    \begin{lstlisting}[language=swift]
                      }
                        .overlay(RoundedRectangle(cornerRadius: 15)
                        .stroke(Color.theme.white, lineWidth: 2)
                        .shadow(color: Color.theme.darkGray.opacity(0.3), radius:4, x:0, y:4))
                        .frame(width: .infinity, height: 160)
                        .padding(.top)

\end{lstlisting}   
\end{minipage}

\section*{További modulok} 
\addcontentsline{toc}{section}{További modulok}

\subsection*{ScannerViewModel modul}
\addcontentsline{toc}{subsection}{ScannerViewModel modul}

\begin{spacing}{2}
\end{spacing}
\begin{minipage}{\textwidth}
    \linespread{0.8}\selectfont
    \begin{lstlisting}[language=swift]
        class ScannerViewModel: ObservableObject {
        
            let scanIterval: Double = 1.0
            
            @Published var torchIsOn: Bool = false
            @Published var lastQrCode: String = "Qr-code goes here"
            
            func onFoundQrCode (_ code: String){
                self.lastQrCode = code
            }
        }
\end{lstlisting}   
\end{minipage}

\subsection*{ViewModel modul}
\addcontentsline{toc}{subsection}{ViewModel modul}

\begin{spacing}{2}
\end{spacing}
\begin{minipage}{\textwidth}
    \linespread{0.8}\selectfont
    \begin{lstlisting}[language=swift]
        final class ViewModel: ObservableObject {
    
            @Published var searchText = ""
            @Published var user: [User] = []
            @Published var traningPlan: [TraningPlan] = []
            @Published var filteredPlan: [TraningPlan] = []
            @Published var exercise: [Exercise] = []
            
            @Published var saveErrorMessage = ""
            
            init(){
                getUserData()
                getExerciseData()
                getTraingPlanData()
            }
            
            private let auth = Auth.auth()
            private let db = Firestore.firestore()
            var uuid: String? {
                auth.currentUser?.uid
            }
            var userIsAuthenticated: Bool {
                auth.currentUser != nil
            }
            var userIsAuthenticatedAndSynced: Bool {
                user != nil && userIsAuthenticated}
            
            func filterContent() {
                   let keywordRegex = "\\b(\\w*" + searchText.lowercased() + "\\w*)\\b"
        
                   if searchText.count > 1 {
                       var matchingPlans: [TraningPlan] = []
                       traningPlan.forEach { traning in
                           let searchContent = traning.name + traning.level
                           if searchContent.lowercased().range(of: keywordRegex, options: .regularExpression) != nil {
                               matchingPlans.append(traning)
                           }
                       }
                       self.filteredPlan = matchingPlans
                   } else {
                       filteredPlan = traningPlan
                   }
               }
    \end{lstlisting}   
\end{minipage}

\begin{spacing}{2}
\end{spacing}
\hspace{-10mm}
\begin{minipage}{\textwidth}
    \linespread{0.8}\selectfont
    \begin{lstlisting}[language=swift]   
            func getUserData() {
                db.collection("users").getDocuments { snapshot, error in
                    if error == nil {
                        if let snapshot = snapshot {
                            DispatchQueue.main.async {
                                self.user = snapshot.documents.map { document in
                                    return User(id: document.documentID,
                                                email: document["email"] as? String ?? "",
                                                password: document["password"] as? String ?? "",
                                                firstName: document["firstName"] as? String ?? "",
                                                lastName: document["lastName"] as? String ?? "",
                                                imageUrl: document["imageUrl"] as? String ?? "")
                                }
                            }
                        }
                    } else {
                        print("Success")
                    }
                }
            }
            func getUserInformation(email: String, lastName: String,firstName: String,password: String, imageUrl: String){
                guard let uid = Auth.auth().currentUser?.uid else {return}
                let userData = ["email": email, "firstname": firstName, "lastname": lastName, "imageurl": imageUrl, "id": uid]
                Firestore.firestore().collection("users").document(uid).setData(userData) { err in
                    if let err = err {
                        print(err)
                        return
                    }
                }
            }
            func addUserData(firstName: String, lastName: String, email: String, password: String) {
                guard let uid = Auth.auth().currentUser?.uid else {return}
                db.collection("users").document(uid).setData(["firstName": firstName, "lastName": lastName, "email": email, "imageUrl": "ImageURL", "password": password]) { error in
                    if error == nil {
                        self.getUserData()
                    }
                    else {
                        print("Success")
                    }
                }
            }
            func getTraingPlanData() {
                db.collection("traningPlans").getDocuments { snapshot, error in
                    if error == nil {
                        if let snapshot = snapshot {
                            DispatchQueue.main.async {
                                self.traningPlan = snapshot.documents.map { document in
                                    return TraningPlan(id: document.documentID,
                                                    name: document["name"] as? String ?? "",
                                                    duration: (document["duration"] as? Int ?? nil)!,
                                                    level: document["level"] as? String ?? "",
                                                    imageUrl: document["imageUrl"] as? String ?? "",
                                                    exercises: self.exercise
                                    )
                                }
                                self.filteredPlan=self.traningPlan
                            }
                        }
                }
            }
            func getExerciseData() {
                db.collection("exercises").getDocuments { snapshot, error in
                    if error == nil {
                        if let snapshot = snapshot {
                            DispatchQueue.main.async { [self] in
                                self.exercise = snapshot.documents.compactMap { document in
                                    return Exercise(id: document.documentID,
                                                    name: document["name"] as? String ?? "",
                                                    setOf: document["setOf"] as? String ?? "",
                                                    imageUrl: document["imageUrl"] as? String ?? "",
                                                    description: document["description"] as? String ?? "",
                                                    machineNumber: (document["machineNumber"] as? Int ?? nil)!
                                    )
                                }
                            }
                            
                        }
                    } else {
                        print("Bad request")
                    }
                }
            }
        }
    
    \end{lstlisting}   
\end{minipage}

\begin{spacing}{2}
\end{spacing}
\hspace*{-10mm}
\begin{minipage}{\textwidth}
    \linespread{0.8}\selectfont
    \begin{lstlisting}[language=swift]
        func updateUser(email: String, password: String, firstName: String, lastName: String) {
                        
                        guard let id = Auth.auth().currentUser?.uid else { return }
                        
                        db.collection("users").document(id).updateData(["email": email,"firstName": firstName,"lastName": lastName, "password": password]){ err in
                            if err != nil{
                                self.saveErrorMessage = err?.localizedDescription ?? ""
                                print((err?.localizedDescription)!)
                                return
                            }
                            print("Success")
                        }
                    }
        }
    \end{lstlisting}   
\end{minipage}

\subsection*{ViewRouter modul}
\addcontentsline{toc}{subsection}{ViewRouter modul}

\begin{spacing}{2}
\end{spacing}
\begin{minipage}{\textwidth}
    \linespread{0.8}\selectfont
    \begin{lstlisting}[language=swift]

        class ViewRouter: ObservableObject {
            
            @Published var currentPage: Page = .welcomePage
        }

        enum Page {
            case welcomePage
            case registerPage
            case logInPage
            case homePage
            case traningPage
            case exercisePage
            case qrPage
        }
    \end{lstlisting}   
\end{minipage}  

\subsection*{ColorTheme modul}
\addcontentsline{toc}{subsection}{ColorTheme modul}

\begin{spacing}{2}
\end{spacing}
\begin{minipage}{\textwidth}
    \linespread{0.8}\selectfont
    \begin{lstlisting}[language=swift]
        extension Color {
            
            static let theme = ColorTheme()
            
        }

        struct ColorTheme {
            
            let darkBlue = Color("DarkBlue")
            let lighBlue = Color("LightBlue")
            let darkGray = Color("DarkGray")
            let gray = Color("Gray")
            let white = Color("White")
            let red = Color("Red")
        }
    \end{lstlisting}   
\end{minipage}

\subsection*{SetFontFamily modul}
\addcontentsline{toc}{subsection}{SetFontFamily modul}

\begin{spacing}{2}
\end{spacing}
\begin{minipage}{\textwidth}
    \linespread{0.8}\selectfont
    \begin{lstlisting}[language=swift]
        extension SetFontFamily{
            static func custom(_ font:CustomFonts,size:CGFloat) -> SwiftUI.Font{
                SwiftUI.Font.custom(font.rawValue, size: size)
            }
        }
        enum CustomFonts: String{
            case regular = "Ubuntu-Regular"
            case italic = "Ubuntu-Italic"
            case bold = "Ubuntu-Bold"
            case medium = "Ubuntu-Medium"
            case light = "Ubuntu-Light"
            case mediumitalic = "Ubuntu-MediumItalic"
        }
}
\end{lstlisting}   
\end{minipage}